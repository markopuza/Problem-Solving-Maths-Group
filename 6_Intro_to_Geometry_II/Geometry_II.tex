\documentclass[11pt,a5paper]{article}
\usepackage[utf8]{inputenc}
\usepackage[english]{babel}
\usepackage{amsmath}
\usepackage{amsthm}
\usepackage{amsfonts}
\usepackage[margin=0.47in]{geometry}
\usepackage{graphicx}

\newtheorem{theorem}{Example}
\newtheorem{exercise}{Exercise}
\newtheorem*{Theorem}{Theorem}

\title{\textbf{Geometry II.}}
\date{Week 9}
\author{Miroslav Stankovic\\ Marko Puza}
\begin{document}
\maketitle

\section{Theory}
After having looked into angle-chasing problem solving approaches in geometry last week, the following pages contain simple tools that will enable us to augment angle-chasing with algebraic tools.\\
More precisely, we will try to reason about angles using lengths and vice versa using Power of a point and Radical lines.

\begin{Theorem}[Power of a point]
Let $s$ denote distance of point $P$ from center $O$ of a circle $k$ with radius $r$. The \emph{Power of a point} with respect to $k$ is a real number $h$ defined as: \[h = s^2 - r^2\]
This number reflects the relative distance of $P$ from $k$. \\
Furthermore, let an arbitrary line through $P$ intersect $k$ in (not necessarily distinct) points $X, Y$. Then \[|PX|\cdot|PY| = h\]
\end{Theorem}

\begin{figure}[h] \begin{center}
\includegraphics[width=0.5\textwidth]{power}
\caption{Here, $|PA|\cdot|PB| = |PU|\cdot|PV| = h$}
\end{center} \end{figure}

\begin{proof}
We will prove only the case where the point lies outside of a circle - other cases can be proved analogically using similar tirangles.
Consider line through $P$ that intersects circle $k$ in points $X, Y$ and point $T$ also on $k$ such that line $PT$ is tangent to circle $k$. We know that $\angle PAT = \angle PTB$ (circumscribed angles) and thus by \emph{uu} triangles $\triangle PAT, \triangle PTB$ are similar. This implies that $\frac{|PA|}{|PT|} = \frac{|PT|}{|PB|}$ or equivalently $|PA|\cdot|PB| = |PT|^2$. However, by the Pythagoras theorem also $|PT|^2 = s^2 - r^2$.
\end{proof}

\begin{exercise}
Prove that Power of a point theorem holds for any $n$-dimensional sphere, $n \ge 2$.
\end{exercise}

\begin{Theorem}
Two circles are orthogonal if they intersect in right angles. Equivalently: \[r_1^2 + r_2^2 = d^2\] where $r_1, r_2$ are radii of the circles and $d$ distance of their centers.
\end{Theorem}

\begin{Theorem}[Chordal theorem]
Given two circles $k, l$ with distinct centers, the set of points such that their power is same with respect to $k$ and with respect to $l$ is a line $R_{k, l}$ perpendicular to the line joining centers of $k, l$. It is known as \emph{radical axis} of the two circles. \\
Furthermore, the radical axis is set of centers of all circles that are orthogonal to both $k, l$.
\end{Theorem}

\begin{figure}[h] \begin{center}
\includegraphics[width=0.5\textwidth]{radical}
\end{center} \end{figure}

\begin{exercise}
Considering the definition of orthogonal circles above, prove that a radical axis is indeed a set of centers of all circles that are orthogonal to $k, l$.
\end{exercise}


\begin{Theorem}[Radical axis theorem]
Let there three circles $a,b,c$ given such that no two are concentric. The three radical axes (for each pair of circles) intersect in a single point called the \emph{radical center}, or are parallel. \\
This also means that there is a unique circle with its center at the radical center that is orthogonal to all three circles. 
\end{Theorem}

\begin{proof}
Consider radical axes $R_{a,b}, R_{b,c}$ intersecting in point $C$. By definition of a radical axis the tangents to $a$ and $b$ through $C$ are equal in length, as well as tangents to $b$ and $c$ through $C$ are. By the transitivity of equality, also tangents to $a$ and $c$ through $C$ are equal in length, which means that $C$ also lies on $R_{a,c}$.
\end{proof}

\begin{figure}[h] \begin{center}
\includegraphics[width=0.5\textwidth]{center}
\end{center} \end{figure}


\begin{exercise}
Prove by similar argument that the unique circle orthogonal to all three circles is indeed centered in radical center.
\end{exercise}

\noindent Let's finish the theory section by mentioning that radical lines play important role in solution of a Problem of Apollonius \footnote{Given three objects, each of which may be a point, line, or circle, draw a circle that is tangent to each}.

\section{Problems}


\begin{enumerate}
\subsection*{Easy}

    \item{Prove that quadrilateral $ABCD$ with $M = AC \cap BD$ is cyclic if and only if $|MA|\cdot|MC| = |MB|\cdot|MD|$.}


\subsection*{Medium}
	
	
\subsection*{Difficult}
	

\end{enumerate}

\begin{thebibliography}{9}
\bibitem{KMS} Ondrej Budáč, Tomáš Jurík, and Ján Mazák. 
	\emph{Zbierka úloh KMS}. Trojsten, Bratislava, 2010.
	
\bibitem{PraSe} Matematický korespondenční seminář. Knihovna. [ONLINE] Available at: \texttt{https://mks.mff.cuni.cz/library/library.php}. [Accessed November 10].
\end{thebibliography}
\end{document}
