\documentclass[12pt]{article}
\usepackage[utf8]{inputenc}
\usepackage[english]{babel}
\usepackage{amsmath}
\usepackage{amsthm}
\usepackage{graphicx}


\newtheorem{theorem}{Example}
\newtheorem{problem}{}

\title{\textbf{Everything}}
\date{Week 10}
%\author{Miroslav Stankovic\\ Marko Puza\\ Ivan Lau\\}
\begin{document}
\maketitle

\section{Invariance Principle}

Property which does not change is called \emph{invariant}, property that only changes in one direction, e.g. always increases, is called \emph{monovariant}. These properties are often used in problems involving different states and transitions between them. 

\begin{problem}
Blag IP \end{problem}

\begin{problem}
Blag IP \end{problem}

\begin{problem}
In an $n \times n$ board the squares are painted black or white. Three of the squares in the corners are white and one is black. Show that there is a $2 \times 2$ square with an odd number of white unit squares.
 \end{problem}

\section{Extreme Principle}

One of basic problem solving strategies is \emph{Extreme principle}. It uses some extreme property of the problem (e.g. largest element, longest path, or smallest sum) and uses this to construct proof or show contradiction. 

Good indicator that Extreme Principle might be useful is when problem has bounded set of possible values, or there is invariant or monovariant.

\begin{problem}
Blag EP \end{problem}

\begin{problem}
All plane sections of a solid are circles. Prove that the solid is a ball.
\end{problem}

\begin{problem}
Blag EP \end{problem}

\section{Pigeonhole Principle}

Also known as \emph{Dirichlet's principle}, it comes from the fact, that if you have more than $mn$ pigeons and $n$ holes, there must be a hole with at least $m+1$ pigeons. It is often used to prove existance of some object (with given property).

\begin{problem}
Blag PP
\end{problem}

\begin{problem}
Show, that in a party there are always two persons who have shaken hands with the same number of people.
\end{problem}

\begin{problem}
We are given $1985$ positive integers such that none has a prime divisor greater than $23$. Show that there are $4$ of them whose product is the fourth power of an integer.
\end{problem}

\section{AG-Inequality}

AG states, that for nonegative real numbers $x_1, x_2, \dots x_n$: 
\[\sqrt[n]{x_1x_2\dots x_n} \le  \frac{x_1 + x_2 + \dots + x_n}{n}\] 
with equality if and only if $x_1 = x_2 = \dots = x_n$.

It can be used to solve many inequality problems, often summing multiple AGs together.

\begin{problem}
Prove that for $x\ge 0$: $1 + x \ge 2\sqrt{x}$.
\end{problem}
\begin{problem}
Prove that for $x\ge 0$: $2 + 3x^5 \ge 5x^3$.
\end{problem}
\begin{problem}
Prove that for possitive $a$, $b$, $c$ with $abc = 1$ and $a+b+c \ge a^{-1} + b^{-1} + c^{-1}$ that also $a^n+b^n+c^n \ge a^{-n} + b^{-n} + c^{-n}$ for every possitive integer $n$.
\end{problem}

\section{Geometry}

One of most useful properties in geometry, is that given points $A$, $B$, $C$ on a circle, angle $\angle ABC$ does not change as $B$ moves along the arc. Consequence of this quality are properties of cyclic quadrilaterals:\\
- Any side is viewed with same angle from any of the remaining vertices.\\
- The opposite angles add up to $180^\circ$.\\
\\One of the methods how to find and make use of cyclic quadraliterals is so-called Power of a point. it states, that for any circle $k$ (centred at $S$ with radius $r$) and point $P$: If line through $P$ intersects $k$ in $A$ and $B$, then $PA \cdot PB$ is always same and equal to $PS^2 - r^2$.

\begin{problem}
Blag GG
\end{problem}

\begin{problem}
Blag GG
\end{problem}

\begin{problem}
Blag GG
\end{problem}

\begin{problem}
Blag GG
\end{problem}

\end{document}
