\documentclass[11pt,a5paper]{article}
\usepackage[utf8]{inputenc}
\usepackage[english]{babel}
\usepackage{amsmath}
\usepackage{amsthm}
\usepackage[margin=0.47in]{geometry}


\newtheorem{theorem}{Example}
\newtheorem{exercise}{Exercise}
\newtheorem*{Theorem}{Theorem}

\title{\textbf{Mean Inequalities}}
\date{Week 5}
\author{Miroslav Stankovic\\ Marko Puza}
\begin{document}
\maketitle

\section{Theory}
\subsection*{Basics}

Inequality problems are a very broad topic. In this text, we will 
mostly focus the inequalities between means and how to apply them 
to solve wide range of inequalities. \\

We will be only concerned with real numbers (often only non-negative 
real numbers). One of basic properties of real numbers is that the 
square of a real number is non-negative, thus so is sum of squares. 
This is the most basic useful inequality.

\begin{Theorem} \emph{Sum of squares inequality.}
If $x_1, x_2, \dots x_n \in \mathbf{R}$ and 
$\alpha_1, \alpha_2, \dots \alpha_n > 0$ then 
$\sum_{i=1}^n \alpha_1x_1^2\ge0$ with equality if and only if 
$x_1 = x_2 = \dots = x_n = 0$.
\end{Theorem}

\begin{theorem} Prove that $x^2 + y^2 + 2 \ge 2x + 2y$ and find when 
there is equality.
\end{theorem}
\begin{proof} 
	\begin{align*} 
		x^2 + y^2 + 2 & \ge 2x + 2y \\
		x^2 -2x +1 + y^2 -2y + 1 & \ge 0 \\
		(x-1)^2 + (y-1)^2 & \ge 0
	\end{align*}
	So our inequality is equivalent to the $(x-1)^2 + (y-1)^2 \ge 0$, 
	which is Sum of squares inequality with equality for $x=y=1$
\end{proof}
\begin{exercise} Prove that $4x^2 + y^2 + 3 \ge 4x + 2y$.
\end{exercise}

Thanks to the work of Emil Artin and Charles Delzell it is now 
known, that it is possible to prove any inequality involving 
rational functions simply by reducing it to a sum of squares. 
However, to do so would be outstandingly tedious and so we 
introduce much more powerful method - Arithmetic-Geometric mean 
inequality.

\subsection*{Arithmetic-Geometric mean inequality}

A well known inequality, often used in problem solving, is an 
inequality between arithmetic and geometric mean. It is often 
called AM-GM inequality or simply AG.

\begin{Theorem} \emph{AG.}
For nonegative real numbers $x_1, x_2, \dots x_n$ : 
\[\sqrt[n]{x_1x_2\dots x_n} \le  \frac{x_1 + x_2 + \dots + x_n}{n}\] 
with equality if and only if $x_1 = x_2 = \dots = x_n$. \\
\end{Theorem}

\noindent We will prove AG for $n=2$. The AG becomes 
$\sqrt{x_1x_2} \le  \frac{x_1 + x_2}{2}$. 
\begin{proof} Using that square of a real is non-negative and basic 
algebraic operations:
	\begin{align*}
	0 & \le (x-y)^2 \\
	 & = x^2 - 2xy + y^2 \\
	 & = x^2 + 2xy - y^2 -4xy \\
	 & = (x+y)^2 - 4xy 
	\end{align*}
	Thus $4xy \le (x+y)^2$ from which 
	$\sqrt{x_1x_2} \le  \frac{x_1 + x_2}{2}$ follows imidiately. 
\end{proof}

\begin{theorem}
	Prove that for possitive real numbers $x, y, z$: 
	$x^3 + y^3 + z^3 \ge x^2y + y^2z + z^2x$.
\end{theorem}

\begin{proof} Let's look at each expression on the right-hand side 
separatelly. We can treat $x^2y$ as product of three numbers 
$x\cdot x\cdot y$ and using AG we have 
\[x^2y = x\cdot x\cdot y = \sqrt[3]{x^3x^3y^3} \le\frac{x^3 + x^3 + y^3}{3}\]
We get very similar inequalities for $y^2z$ and $z^2x$ - summing 
those three up we get exactly the inequality we wanted to prove.
\end{proof}

\begin{exercise} Prove that for possitive real numbers $x, y, z$: 
$x^4 + y^4 + z^4 \ge x^2yz + xy^2z + xyz^2$.
\end{exercise}

\subsection*{Power Mean Inequality}

\section{Problems}

\begin{enumerate}
	\subsection*{Easy}

	\subsection*{Medium}
	
	\subsection*{Difficult}

	\subsection*{Extra}	
	\item{Prove AG using induction.}
	
	\item{Prove that power mean tends to geometric mean as the power 
	tends to zero.}
\end{enumerate}

\begin{thebibliography}{9}
\bibitem{KMS} Ondrej Budáč, Tomáš Jurík, and Ján Mazák. 
	\emph{Zbierka úloh KMS}. Trojsten, Bratislava, 2010.

\end{thebibliography}

\section{Hints}
\begin{enumerate}
	\subsection*{Easy}

	\subsection*{Medium}
	
	\subsection*{Difficult}

	\subsection*{Extra}	
	\item{Use the three step induction. Show that if AG holds for $n$ 
	numbers then it also holds for $2n$ numbers and then that if it 
	holds for $m$ numbers it also holds for $2m-1$ numbers. (Why this 
	induction works?)}
	
	\item{No hint for this one.}
\end{enumerate}

\end{document}

