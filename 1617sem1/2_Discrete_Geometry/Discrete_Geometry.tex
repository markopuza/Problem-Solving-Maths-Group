\documentclass[12pt]{article}
\usepackage[utf8]{inputenc}
\usepackage[english]{babel}
\usepackage{amsmath}
\usepackage{amsthm}
\usepackage{graphicx}

\graphicspath{ {/home/marko/Problem solving maths group} }
\newtheorem{theorem}{Theorem}
\newtheorem{definition}{Definition}
\newtheorem{example}{Example}
\newtheorem{exercise}{Exercise}

\title{\textbf{Discrete Geometry 1}\\Pick's Theorem}
\date{Week 2}
\author{Miroslav Stankovic\\ Marko Puza}
\begin{document}
\maketitle

\section{Intro}

\emph{Discrete Geometry} is a branch of mathematcs dealing with combinations and arrangements of geometric objects and discrete properties of such objects. Questions in discrete geometry usualy involve countable sets of geometrical objects, such as points, lines, polygons. Here we will introduce one of the significant theorems in this area - Pick's Theorem. 


\section{Pick's Theorem}

The theorem deals with geometry on a grid of equal-distanced points (i.e., points with integer coordinates). It states, that area $A$ of a simple polygon (one without holes) with vertices on grid points is $i + \frac{b}{2} - 1$, where $i$ is number of grid points inside the polygon and $b$ number of grid points on the polygon boundary.


\begin{exercise}
	Show that the formula holds for rectangles.
\end{exercise}

\begin{exercise}
	Show that the formula holds for right-agled triangles.
\end{exercise}

\begin{exercise}
	Show that the formula holds for all triangles.
\end{exercise}

\begin{exercise}
	Show that if the formula holds for two polygons which share a segment of perimeter, it also holds for the union of the two.
\end{exercise}

\begin{exercise}[Pick's Theorem] 
	Show that formula holds for any simple polygon.
	For any simple polygon: $A = i + \frac{b}{2} - 1$.
\end{exercise}

\begin{exercise}
	Let's have a grid with length of side $\sqrt{2}$. Show that every polygon with vertices in grid points has an integral area.
\end{exercise}

\begin{exercise}
	Does the formula change for polygons with holes?
\end{exercise}

\begin{exercise}
	Can the theory be extended to more dimensions? Is there similar formula if we look at triangular grid instead?
\end{exercise}


\end{document}

