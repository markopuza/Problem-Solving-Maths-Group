\documentclass[11pt,a5paper]{article}
\usepackage[utf8]{inputenc}
\usepackage[english]{babel}
\usepackage{amsmath}
\usepackage{amsthm}
\usepackage[margin=0.47in]{geometry}


\newtheorem{theorem}{Example}

\title{\textbf{Pigeonhole Principle}}
\date{Week 4}
\author{Miroslav Stankovic\\ Marko Puza}
\begin{document}
\maketitle

\section{Theory}
\subsection*{Invariants}

The \emph{Invariants} (also known as \emph{The Invariance Principle}) is an extremely useful problem solving technique. Problems where this strategy can be used are often easily recognizable and the tricky part is usually to find a suitable invariant. It is frequently used to solve probelms involving algorithms (transformations, games) where some task is repeatedly performed.  \textbf{When there is repetition, look for what does not change!}

In algorithm problems, there is usually a starting state $S$ and a sequence of legal actions (steps, moves). The question tend to be one of following: \\
$1$. Can a given state be reached from S?\\
$2$. Find all states reachable from S.\\
$3$. Is there a convergence to an end state?\\
$4$. Can we return to the starting state S after given number of moves?\\\\\\\\\\

\begin{theorem}
	Lorem ipsum talor ornae, et learom otna demoner a dulis?
\end{theorem}
\begin{proof}
	Blah. 
\end{proof}

\section{Problems}

\begin{enumerate}
	\subsection*{Easy}	
	
	\subsection*{Medium}
	\item{Problem.}

	\subsection*{Difficult}
	
	
\end{enumerate}

\begin{thebibliography}{9}
\bibitem{IMO Math} Duke Putnam Preparation. 2012. [ONLINE] Available at: \texttt{http://www.imomath.com/index.php?options=526\&lmm=0}. [Accessed 10 October 15].

\end{thebibliography}

\end{document}
http://bayanbox.ir/view/3155187039802412539/Problem-Solving-Strategies-Arthur-Engel.pdf

