\documentclass[12pt]{article}
\usepackage[utf8]{inputenc}
\usepackage[english]{babel}
\usepackage{amsmath}
\usepackage{amsthm}
\usepackage{graphicx}

\graphicspath{ {/home/marko/Problem solving maths group} }
\newtheorem{theorem}{Example}

\title{\textbf{Extreme principle}}
\date{Week 3}
\author{Miroslav Stankovic\\ Marko Puza}
\begin{document}
\maketitle

\section{Solutions}

\begin{enumerate}
	\item{Take the smallest triangle. Some two of it's vertices are of the same color, thus there is a point on the side. Using this point, we can find smaller triangle - we have a contradiction.
	}
	
	\item{Look at smallest number, $n$. All it's neighbors are at least $n$, to have arithmetic mean $n$ they must be all equl to $n$.}
	
	\item{Suppose we have joined points in such a way, that combined length of line segments is minimal. Now if two lines intersect, we can easily find a contradiction.}
	
	\item{The cells containing $1$ and $n^2$ are at most $n-1$ steps apart, while the diffenence is $n^2-1$. If no two adjacent cells deffered by more than $n$, the difference between $1$ and $n^2$ could be at most $n(n-1)$ - contradiction.}
	
	\item{Take such five sheets, that, when removed, decrease the covered area the least. Let $A$ be area covered only be some of these sheets. Divide rest of the sheets into two quintets and let $B$, $C$ be area covered only by sheets from a quintet. Let $X$ be area covered by sheets from multiple quintets. Having $A\le B$, $A\le C$ and $A+B+C+X=1$ it is easy to show that $B+C+X \ge 2/3$.}
	
	\item{Take a line $p$ and point $A$, such that distance $d$ between $A$ and $p$ is minimal nonzero. From three points on $p$, some two form with $A$ an obtuse triangle $ABC$ with obtuse angle at point $B$ or $C$ - wlog let be it at $B$. Then distance between $B$ and line $AC$ is less then $d$, which is contradiction.}
	
	\item{}

	\item{(a) If $p$ has a minimum at some point $a$, we have $p'(a)=0$ thus $p(a)\ge 0$ and so $p(x)\ge 0 $ for all real $x$. If $p$ does not have a minumum, it is of odd degree - in which case $p(x) + p'(x) \ge 0$ does not hold.
\\(b) It does. Function $-e^{-x}$ would be an example.}
	
	\item{}
	
	\item{}


\end{enumerate}
\end{document}

Bonus:
All plane sections of a solid are circles. Prove that the solid is a ball.

Bonus 2:
$2014$ vectors are drawn on a plane. Two players alternately take a vector until there are no more vectors left. The winner is the player whose vectors’ sum is longer. Suggest a winning (or at least a non-losing) strategy for the first player.

Sources:
https://web.viu.ca/bigelow2/Math360ClassProblemsInvolvingExtremePrinciple.pdf
https://www.math.hmc.edu/~ajb/PCMI/pcmi10_b.pdf
https://brilliant.org/wiki/extremal-principle/
http://www.artofproblemsolving.com/wiki/index.php/Extreme_principle
https://www.math.wisc.edu/wiki/images/Putnam112013.pdf


