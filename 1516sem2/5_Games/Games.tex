\documentclass[11pt,a5paper]{article}
\usepackage[utf8]{inputenc}
\usepackage[english]{babel}
\usepackage{amsmath}
\usepackage{amsthm}
\usepackage{amsfonts}
\usepackage[margin=0.47in]{geometry}
\usepackage{graphicx}


\newtheorem{theorem}{Example}
\newtheorem{exercise}{Exercise}
\newtheorem*{Theorem}{Theorem}

\title{\textbf{Games}}
\date{Semester 2 - Week 9}
\author{Jonas Wolter}

\begin{document}
\maketitle

\noindent
Another topic that appears occasionally in mathematical competions and which is also a lot of fun is the topic of \emph{Games}. 
Games in the mathematical context are always seen as a competition between \textbf{two} players who are allowed to conduct different operations. In this session we will only look at games which don't contain any randomness in them (such as rolling a die).\\
The main question that appears asks whether there is a strategy for one player to always win the game or whether the game necesserarily leads to a draw. Note that it cannot happen that both players have the chance to win if we assume that both of them don't make any mistakes within there game. Hence you could say that mathematical games are usually really boring ;-)\\

\noindent
The first game we will consider is the so called \emph{Nim-Game}. Try it out and work out a strategy.\\
\textbf{Starting situation:} There are 20 matches lying on a table.\\
\textbf{Rules:} A player can always take 1,2,3,4 or 5 matches away.\\
\textbf{End of the Game:} The player who takes the last match wins the game.\\
\noindent
Player A begins. Who can win the game?\\
\noindent 
How does the situation change if we start with 30 instead of 20 matches?\\

\noindent
The main question is how to tackle a problem like this if one have no idea. Of course the first and really nice step is to try the game on yourself to see from which situations you can win. In general it is always a good approach to start with very simple situations from which you know which player will win. In the next step one is looking for situation from which one can reach the previous ones where the result is known and hence can deduce which player wins these. Eventually we try to generalise the conditions that are needed for one player to win the game and try to deduce a result for the given task. This idea of constructing a strategy from very simple conditions is often known as \emph{recursion}.\\

\noindent
\textbf{Task 1:}\\
\noindent
Let's look at the general \emph{ Nim - Game} now:\\
\textbf{Starting situation:} There are M matches lying on a table.\\
\textbf{Rules:} A player can always take $1,2,3,...,n$  matches.\\
\textbf{End of the Game:} The player who takes the last match wins the game.\\
Which conditions must held to such that one of the players has a strategy to win the game.\\

\noindent
\textbf{Task 2:}\\
\noindent
Here are some further variations of the \emph{Nim - Game} (Starting situation and Ende of the Game remain the same).\\
\textbf{Rules:}\\[-0.5cm]
\begin{itemize}
\item{A player can always take a number of matches which is a power of 2.}
\item{A player can always take  a number of matches which is 1 or a prime.}
\item{A player can always take 1,3 or 8 matches.}
\end{itemize}
You can investigate this even further and actually find a general strategy that even works if the \emph{Nim-Game} is played with different piles of matches but let's move on to some different games which are of a similar kind.\\

\noindent
\textbf{Task 3:}(\emph{Wythoff's Game})\\
\noindent
\textbf{Starting situation:} There are two piles of  matches lying on a table.\\
\textbf{Rules:} A player can always take any number of matches from one pile or the same number of matches from each pile.\\
\textbf{End of the Game:} The player who takes the last match wins the game.\\
\noindent
Player A begins. Try to find out from which starting positions A can win the game.\\

\noindent
\textbf{Task 4:}\\
\noindent
\textbf{Starting situation:} There are $10^7$ matches on the table.\\
\textbf{Rules:} A player can always take any number of matches which is of the form $p^n$ with $p$ prime and $n$ a positive integer.\\
\textbf{End of the Game:} The player who takes the last match wins the game.\\
\noindent
Player A begins. Who can win the game?\\
Of course there are not only games of this kind so, thus we will investigate some which require different approaches.\\

\noindent
\textbf{Task 4:}\\
\noindent
\textbf{Starting situation:} The integers $1 \quad 2 \quad 3 \quad... \quad 100$ are written on the blackboard such that there is a space between any two consecutive numbers.\\
\textbf{Rules:} A player can always place a $+, -$ or $\times$ in any gap that hasn't been filled.\\
\textbf{End of the Game:} The game ends if every space is filled.\\
\noindent
Player A begins. He wins if:\\
(a) The result on the blackboard is odd.\\
(b) The result on the blackboard is even.\\
Can A win the game?\\


\noindent
\textbf{Task 5:}\\
\noindent
\textbf{Starting situation:} $n=2$ .\\
\textbf{Rules:} A player can always add a divisor of the current $n$ to it.\\
\textbf{End of the Game:} The game ends if n exceeds 1990. The player who did the last move wins.\\
\noindent
Player A begins. Who can win the game?\\

\noindent
\textbf{Task 6:}\\
\noindent
\textbf{Starting situation:} We are given a triangle $ABC$ withf area 1. \\
\textbf{Rules:} Player A chooses a point $P$ on the edge $AB$, then B chooses a point $Q$ on $BC$ and in the end A chooses a point $R$ on $CA$. A tries to make the area of the triangle $PQR$ is big as possible while B aims the opposite.\\
\textbf{End of the Game:} A wins if the area of $PQR$ is greater $\frac{1}{2}$ and otherwise B wins.\\
\noindent
 Who can win the game?\\
How does the situation change if we are playing in a quadrangle or a pentagon?\\

\noindent
\textbf{Task 7:}\\
\noindent
\textbf{Starting situation:} $p=1$ .\\
\textbf{Rules:} A player can always multiply the given $p$ by any integers from 2 to 9.\\
\textbf{End of the Game:} The game ends if $n$ exceeds 1000 ($10^6$). The player who did the last move wins.\\
\noindent
Player A begins. Who can win the game?\\

\noindent
\textbf{Task 8:}\\
\noindent
\textbf{Starting situation:} We are given a regular $1988-gon$.\\
\textbf{Rules:} A player can always connect to vertices if the diagonal doesn't intersect with any line drawn earlier.\\
\textbf{End of the Game:} The game ends if no more lines can be drawn. The one who did the last move wins the game.\\
\noindent
Player A begins. Who can win the game?\\

\noindent
\textbf{Task 9:}\\
\noindent
\textbf{Starting situation:} We are given the polynomial $x^4+*x^3+*x^2*+*x+*$.\\
\textbf{Rules:} A player can always replace a star by an integer of his choice.\\
\textbf{End of the Game:} A wins the game if the polynomial has in integral root. B wins otherwise.\\
\noindent
Player A begins. Who can win the game?\\

\noindent
\textbf{Task 10:}\\
\noindent
\textbf{Starting situation:} We are given a $19 \times 94$ grid.\\
\textbf{Rules:} A player can always colour any square within the the grid ($1\times1, 2\times2, 3\times3,...$).\\
\textbf{End of the Game:} The games end if no more moves are possible. The player who did the last move wins\\
\noindent
Player A begins. Who can win the game?\\
Try to generalise this game for arbitrary side lengths of the grid. \\[1cm]

\begin{center}
Have fun :-)
\end{center}

\end {document}